\documentclass[conference]{IEEEtran}
\usepackage{times}

% numbers option provides compact numerical references in the text. 
\usepackage[numbers]{natbib}
\usepackage{multicol}
\usepackage[bookmarks=true]{hyperref}
\usepackage{amsmath,amssymb}
\usepackage{amsthm}
\usepackage{graphicx}
% \usepackage{caption}
% \usepackage{subcaption}
% \usepackage{subfig}
% \usepackage{float}
\usepackage{tikz}
\usetikzlibrary{scopes}
\usetikzlibrary{shapes.misc}
\tikzset{cross/.style={cross out, draw=black, minimum size=2*(#1-\pgflinewidth), inner sep=0pt, outer sep=0pt},
%default radius will be 1pt. 
cross/.default={2pt}}

\newtheorem{theorem}{Theorem}
\newtheorem{proposition}{Proposition}
\newcommand\numberthis{\addtocounter{equation}{1}\tag{\theequation}}

\newcommand{\EH}[1]{{\color{blue} {Eric: {#1}}  }}

\pdfinfo{
   /Author (Eric Huang)
   /Title  (Robots: Our new overlords)
   /CreationDate (D:20161016120000)
   /Subject (Robots)
   /Keywords (Robots;Overlords)
}

\begin{document}

% paper title
\title{Robotic Pulling}

% You will get a Paper-ID when submitting a pdf file to the conference system
% \author{Author Names Omitted for Anonymous Review. Paper-ID [add your ID here]}

%\author{\authorblockN{Michael Shell}
%\authorblockA{School of Electrical and\\Computer Engineering\\
%Georgia Institute of Technology\\
%Atlanta, Georgia 30332--0250\\
%Email: mshell@ece.gatech.edu}
%\and
%\authorblockN{Homer Simpson}
%\authorblockA{Twentieth Century Fox\\
%Springfield, USA\\
%Email: homer@thesimpsons.com}
%\and
%\authorblockN{James Kirk\\ and Montgomery Scott}
%\authorblockA{Starfleet Academy\\
%San Francisco, California 96678-2391\\
%Telephone: (800) 555--1212\\
%Fax: (888) 555--1212}}


% avoiding spaces at the end of the author lines is not a problem with
% conference papers because we don't use \thanks or \IEEEmembership


% for over three affiliations, or if they all won't fit within the width
% of the page, use this alternative format:
% 
%\author{\authorblockN{Michael Shell\authorrefmark{1},
%Homer Simpson\authorrefmark{2},
%James Kirk\authorrefmark{3}, 
%Montgomery Scott\authorrefmark{3} and
%Eldon Tyrell\authorrefmark{4}}
%\authorblockA{\authorrefmark{1}School of Electrical and Computer Engineering\\
%Georgia Institute of Technology,
%Atlanta, Georgia 30332--0250\\ Email: mshell@ece.gatech.edu}
%\authorblockA{\authorrefmark{2}Twentieth Century Fox, Springfield, USA\\
%Email: homer@thesimpsons.com}
%\authorblockA{\authorrefmark{3}Starfleet Academy, San Francisco, California 96678-2391\\
%Telephone: (800) 555--1212, Fax: (888) 555--1212}
%\authorblockA{\authorrefmark{4}Tyrell Inc., 123 Replicant Street, Los Angeles, California 90210--4321}}


\maketitle

% \begin{abstract}
% \end{abstract}

\IEEEpeerreviewmaketitle

% \section{Introduction}

\section{Theory}

% Press-pull stability theorem figure.
\begin{figure}
  \centering
  \def\iangle{35} % Angle of the inclined plane
  \begin{tikzpicture}[
    scale=1.25, every node/.style={scale=1.25},
    force/.style={>=latex,draw=black,fill=black},
    axis/.style={densely dashed,draw=gray,font=\small},
    ]
    \fill[draw=black,fill=blue!10,thin,rotate=\iangle] (-0.3,-0.5) rectangle (2.3,.5);
    \draw[rotate=\iangle] (1,0) circle[radius=2.4pt] node[cross] {};
    \draw[rotate=\iangle] (1,0) node[above right] {\tiny CoP};
    {[axis,->]
      \draw (0,0) -- (2.5,0) node[right] {$x$};
      \draw (0,0) -- (0,2.5) node[above] {$y$};
    }
    {[force,->]
      \draw (0,0) -- ++(0,1.5) node[right] {$\mathbf{v}_c$};
    }
    \fill (0,0) circle [radius=1.pt];
    \fill (1.2207,0) circle [radius=1.pt] node[below right] {$x_{\text{\tiny IC}}$};
  \end{tikzpicture}
  \caption{Motion of a press-pulled slider.}
  \label{fig:presspull-motion}
\end{figure}

% \begin{figure}[t!]
%   \centering
%   \includegraphics[width=0.8\columnwidth]{fig/1}
%   \caption{H}
%   \label{fig:presspull-ex}
% \end{figure}

\begin{theorem}
  For press-pulling of a rigid body in the plane, the line from the
  press-pull contact point to the body's center of pressure converges
  to the line of motion.
\end{theorem}

\begin{proof}
  Let the $y$-axis of the coordinate frame be aligned with the line of
  motion of the press-pull contact point (see Figure
  \ref{fig:presspull-motion} for an example). Then the instantaneous
  rotation center of the body must fall on the $x$-axis, and we can
  write $\mathbf{r}_\text{IC} = (x_\text{IC},0)^T$.

  Suppose the center of pressure is strictly to the right of the line
  of motion (as in Figure \ref{fig:presspull-motion}). Then by Theorem
  7.4 in \cite{Mason}, the rotation center lies on the positive
  $x$-axis and has negative rotation. The velocity of a point on the
  rigid body is given by
  $\mathbf{v}(\mathbf{r}) =
  \dot{\theta}\,\hat{\mathbf{k}}\times(\mathbf{r} -
  \mathbf{r}_\text{IC})$.
  We can compute the motion of the body relative to the contact point
  by
  % 
  \begin{align*}
    \mathbf{v}(\mathbf{r}) - \mathbf{v}_c &= \dot{\theta}\,\hat{\mathbf{k}}\times(\mathbf{r} - \mathbf{r}_\text{IC}) - \dot{\theta}\,\hat{\mathbf{k}}\times(\mathbf{0} - \mathbf{r}_\text{IC}) \\
    &= \dot{\theta}\,\hat{\mathbf{k}}\times(\mathbf{r} - \mathbf{0}) \numberthis \label{eqn:rel-rotation-center},
  \end{align*}
  % 
  where $\mathbf{v}_c$ is the velocity of the contact point. In other
  words, the body is rotating clockwise relative to the
  contact point with angular velocity
  % 
  \begin{equation}
    \dot{\theta} = -\frac{\lVert\mathbf{v}_c\rVert}{x_\text{IC}}.
  \end{equation}
  %

  Let $\theta \in (\pi/2, -\pi/2)$ be the orientation of the center of
  pressure in the frame of the contact point. When the line of motion
  passes through the center of pressure, a pure translation takes
  place (Theorem 7.4 in \cite{Mason}). This implies that
  $\dot{\theta} = 0$ if and only if $\theta = \pi/2, -\pi/2$. Since
  $\theta$ is monotonically decreasing, we see that it must converge
  to $-\pi/2$ in the limit as $t \rightarrow \infty$.

  The case when the center of pressure is strictly to the left of the
  line of motion follows from symmetry about the $y$-axis.
\end{proof}

Note that the above proof holds for \textit{all pressure
  distributions} of the body on the support
surface. % Given that pressure
% distributions are highly indeterminate, this is a very strong result.

% \begin{proposition}
%   When pressing down on a point contact on a rigid body, the center of
%   pressure is given by
%   \begin{equation}
%     \text{CoP} = \frac{1}{f_0 + p_c}\left(\int_R\mathbf{r}p(\mathbf{r})dA + \mathbf{r}_cp_c\right)
%   \end{equation}
% \end{proposition}
% In other words, the resulting center of pressure is located on the
% line segment between the contact point and the original center of
% pressure. 

% \begin{proposition}
%   A rigid body in the plane can be moved from A to B with arbitrary
%   accuracy using at most two press-pulls.
% \end{proposition}

\bibliographystyle{plainnat}
\bibliography{references}

\end{document}


